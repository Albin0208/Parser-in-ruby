\documentclass{TDP003mall}

\newcommand{\version}{Version 1.2}
\author{Albin Dahlén, \url{albda746@student.liu.se}\\
Filip Ingvarsson, \url{filin764@student.liu.se}}
\title{Implementeringsplan}
\date{2023-02-27}
\rhead{Albin Dahlén \\
Filip Ingvarsson}

\begin{document}

    \projectpage

    \tableofcontents

    \section*{Revisionshistorik}
    \begin{table}[!h]
        \begin{tabularx}{\linewidth}{|l|X|l|}
            \hline
            Ver. & Revisionsbeskrivning      & Datum    \\\hline
            1  & Implementeringsplan     & 23-02-22 \\\hline
        \end{tabularx}
    \end{table}

    \newpage


    \section{Översikt}

    \subsection{Planeringsfas}
    Under denna fas kommer vi att planera vilka delar som ska utvecklas.
    Vi kommer även skapa en BNF över språket och speca upp vilken funktionalitet språket kommer ha.
    
    Under denna fasen utvecklades även parser som sedan kommer att byggas på under implementeringsfasen.

    \subsection{Implementeringsfas}
    I denna fasen kommer vi att utveckla de olika delarna av språket.
    Vi kommer att börja med de minsta delarna så som aritmetik och logik och jobba oss uppåt mot klasser.
    Innan vi går vidare kommer vi se till att vi har testat utförligt för att inte stöta på oväntade problem senare när vi sätter ihop funktionaliteten till exempelvis klasser.

    \subsection{Dokumentationsfas}
    I denna fasen kommer dokumentation för all kod att sammanställas.

    \section{Parser}
    Parser som kommer användas är en egenutvecklad parser.

    \section{Backlog}
    \subsection{Aritmetik}
    Implementera +, -, *, /, \%
    Implementera prioriterings ordning med ()
    \subsection{Logik}
    Implementera <, >, >=, <=, ==, !=, \&\&, ||
    \subsection{Variabler}
    Implementera Variabler, int, float, string, bool.
    \subsection{If-else-satser}
    Implementera If else.
    \subsection{Elsif-satser}
    Implementera elsif till if-satserna
    \subsection{Scope-hantering}
    Implementera scope
    \subsection{Containers}
    Implementera containers (arrayer och hash-maps).
    Arrayer: int[], float[], string[], bool[].
    Hash-maps: Hash<key\_type, value\_type>. Hash<string, int>
    \subsection{Loopar}
    Implementera for-loopar, while-loopar och loopar över containrar.
    \subsection{Funktioner}
    Implementera funktioner
    \subsection{Klasser}
    Implementera klasser
    \subsection{Arv till klasser}
    Implementera arv till klasser
    
    
    \section{Milstolpar}
    \begin{itemize}
    \item Aritmetik
    \item Logik
    \item Variabler
    \item If-else-satser
    \item Elsif-satser
    \item Scope-hantering
    \item Containers
    \item Loopar
    \item Funktioner
    \item Klasser
    \end{itemize}

\end{document}
